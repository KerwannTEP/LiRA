% Created 2021-01-13 mer. 22:12
% Intended LaTeX compiler: pdflatex
\documentclass[11pt]{article}
\usepackage[utf8]{inputenc}
\usepackage[T1]{fontenc}

\usepackage[a4paper,bindingoffset=0.5in,%
            left=0.5in,right=0.5in,top=1in,bottom=1in,%
            footskip=.25in]{geometry}

\usepackage{graphicx}
\usepackage{amsmath}
\usepackage{amssymb}
\usepackage{xspace}
\usepackage{commath}
\usepackage{times}
\usepackage{bm} 
\usepackage{balance}
\usepackage{hyperref}
\usepackage{mathtools}
\usepackage{stmaryrd}
\usepackage{wasysym}

\usepackage[toc,page]{appendix}

%%%% Define Acronyms
\usepackage{acronym}
%
\newacro{DF}{distribution fonction}

\newcommand{\bv}{\boldsymbol{v}}
\newcommand{\bnab}{\boldsymbol{\nabla}}
\newcommand{\rd}{\mathrm{d}}
\newcommand{\vr}{v_{r}}
\newcommand{\vt}{v_{\theta}}
\newcommand{\Sigmad}{\Sigma_{\rm{disk}}}
\newcommand{\Sigmadh}{\Sigma_{\rm{DH}}}
\newcommand{\Phib}{\Phi_{\rm{bulb}}}
\newcommand{\Phid}{\Phi_{\rm{disk}}}
\newcommand{\Phidh}{\Phi_{\rm{DH}}}
\newcommand{\Phisg}{\Phi_{\rm{sg}}}

\newcommand{\Mb}{M_{\rm{bulb}}}
\newcommand{\Md}{M_{\rm{disk}}}
\newcommand{\Mdh}{M_{\rm{DH}}}

\newcommand{\anm}{a_n^m}
\newcommand{\bnm}{b_n^m}
\newcommand{\cnm}{c_n^m}
\newcommand{\alm}{a_l^m}
\newcommand{\blm}{b_l^m}
\newcommand{\clm}{c_l^m}
\newcommand{\Pnm}{P_n^{|m|}}
\newcommand{\Plm}{P_l^{|m|}}
\newcommand{\hPnm}{\widehat{\Pnm}}
\newcommand{\hPlm}{\widehat{\Plm}}
\newcommand{\hI}{\widehat{I}}
\newcommand{\hJ}{\widehat{J}}
\newcommand{\hIp}{\widehat{I'}}
\newcommand{\hJp}{\widehat{J'}}

\newcommand{\homega}{\widehat{\omega}}
\newcommand{\hOmega}{\widehat{\Omega}}
\newcommand{\Omegaref}{\Omega_{\mathrm{ref}}}
\newcommand{\halpha}{\widehat{\alpha}}

\newcommand{\Sigmaref}{\Sigma_{\mathrm{ref}}}
\newcommand{\hSigma}{\widehat{\Sigma}}


\author{Kerwann}
\date{\today}
\title{Notes}
\hypersetup{
 pdfauthor={Kerwann},
 pdftitle={Notes},
 pdfkeywords={},
 pdfsubject={},
 pdfcreator={Emacs 27.1 (Org mode 9.3)}, 
 pdflang={English}}
\begin{document}

\maketitle

\tableofcontents


\section{Self-gravitating thin disk with bulb and dark halo}
\label{sec:sg_disk}

Consider a self gravitating disk with surface density $\Sigma$ and gravitational potential $\Phi$. Let $\bv$ be its velocity field and $P$ its pression field. Then it is described by the system
\begin{align}
&\frac{\partial \Sigma}{\partial t} + \bnab \cdot (\Sigma \bv) = 0 ,\\
&\frac{\partial \bv}{\partial t} + (\bv \cdot \bnab)(\bv) = -\frac{1}{\Sigma} \bnab P - \bnab \Phi ,\\
& \Delta \Phi = 4\pi G \Sigma \delta (z).
\end{align}
Here, $\Phi =\Phib+  \Phid +  \Phidh=\Phib +  \Phisg$. Furthermore, $\Phib$  is such that it doesn't yield any contribution to the surface density, i.e. $\Delta \Phib=0$, nor the pressure field, hence $\Sigma =\Sigmad+\Sigmadh$ and $P =P_{\rm{disk}}+P_{\rm{DH}}$. Letting $q=\Sigmad/\Sigma$, we have

\begin{equation}
\Sigma=\Sigmad+\Sigma_{\rm{DH}}=\Sigmad\bigg(1+\frac{\Sigmadh}{\Sigmad}\bigg) =\frac{1}{q} \Sigmad.
\end{equation}

Suppose that we have a polytrope gas such that $P=\kappa \Sigma^{\Gamma}$ and let
\begin{equation}
\psi = \int \frac{\rd P(\Sigma)}{\Sigma} \Leftrightarrow \bnab \psi = \frac{\bnab P}{\Sigma}.
\end{equation}
 Then
 \begin{equation}
\psi = \frac{\kappa \Gamma}{\Gamma-1} \Sigma^{\Gamma-1}.
\end{equation}
 
 Letting $\Psi = \Phi + \psi$, we obtain
\begin{equation}
\frac{\partial \bv}{\partial t} + (\bv \cdot \bnab)(\bv) = - \bnab \Psi.
\end{equation}

If $\Sigma_{\rm{disk}}\propto\Sigma_{\rm{DH}}$ then $q\in [0,1]$ is constant and this system of equation can be rewritten as 
\begin{align}
&\frac{\partial \Sigmad}{\partial t} + \frac{1}{r} \frac{\partial(r\Sigmad \vr)}{\partial r} + \frac{1}{r} \frac{\partial(\Sigmad \vt)}{\partial \theta} = 0 ,\\
&\frac{\partial \vr}{\partial t} + \vr \frac{\partial \vr}{\partial r} + \frac{\vt}{r} \frac{\partial \vr}{\partial \theta} - \frac{(\vt)^2}{r} = -\frac{\partial \Psi}{\partial r} ,\\
&\frac{\partial \vt}{\partial t} + \vr \frac{\partial \vt}{\partial r} + \frac{\vt}{r} \frac{\partial \vt}{\partial \theta} + \frac{\vr\vt}{r} = -\frac{1}{r}\frac{\partial \Psi}{\partial \theta} ,\\
& \Delta \Phisg = 4\pi G \frac{1}{q} \Sigmad \delta (z).
\end{align}

Let $M$ be the total mass of the galactic disk+bulb. Let $\Phib = - G M (1-x)/\sqrt{c^2+r^2}$ with $x\in [0,1]$. An equilibrium state is given by the Plummer equilibrium, such that,  letting $\xi = (r^2-a^2)/(r^2+a^2) $, that is, $r/a=\sqrt{(1+\xi)/(1-\xi)}$,
\begin{align*}
&\vr^0 = 0,\quad \quad (\vt^0)^2 = r \frac{\partial \Psi^0}{\partial r}, \\
& \psi^0 = \frac{\kappa \Gamma}{\Gamma-1} (\Sigma^0)^{\Gamma-1} = \frac{\kappa \Gamma}{(\Gamma-1)q^{\Gamma-1}} (\Sigmad^0)^{\Gamma-1},\\
& \Sigmad^0 = \frac{xM}{2\pi a^2} \frac{1}{(1+(r/a)^2)^{3/2}} =\frac{xM}{2\pi a^2}   \bigg(\frac{1-\xi}{2}\bigg)^{3/2},\\
& \Phisg^0 = -\frac{G M x}{a q} \frac{1}{\sqrt{1+(r/a)^2}} = -\frac{G M x}{a q} \bigg(\frac{1-\xi}{2}\bigg)^{1/2} .
\end{align*}


Therefore, letting  $$\varepsilon = \frac{U(x=1,q=1)}{|E(x=1,q=1)|} =\frac{3a \kappa \Gamma}{GM} \bigg(\frac{M}{2\pi a^2}\bigg)^{\Gamma-1},$$  we obtain
\begin{align*}
\Psi^0 &=  - \frac{G M (1-x)}{\sqrt{c^2+r^2}}-\frac{G M x}{a q}\bigg(\frac{1-\xi}{2}\bigg)^{1/2} + \frac{\kappa \Gamma}{(\Gamma-1)} (\Sigmad^0/q)^{\Gamma-1} ,\\
{}&=  -\frac{G M (1-x)}{\sqrt{c^2+r^2}}+\frac{G M }{a} \bigg[-(x/q)\bigg(\frac{1-\xi}{2}\bigg)^{1/2} + \frac{\varepsilon(x/q)^{\Gamma-1}}{3(\Gamma-1)}\bigg(\frac{1-\xi}{2}\bigg)^{3(\Gamma-1)/2} \bigg] ,\\
&=  \frac{G Mx}{aq}\bigg[-\frac{a}{c}\frac{q (1-x)}{x}\frac{1}{ \sqrt{1+(r/c)^2}} -\bigg(\frac{1-\xi}{2}\bigg)^{1/2} + \frac{\varepsilon(x/q)^{\Gamma-2}}{3(\Gamma-1)}\bigg(\frac{1-\xi}{2}\bigg)^{3(\Gamma-1)/2} \bigg],\\
(\vt^0)^2 &= r \frac{\partial \Psi^0}{\partial r} =r \frac{\partial \Psi^0}{\partial \xi} \frac{\partial \xi}
{\partial r} = 4\bigg(\frac{r}{a}\bigg)^2    \bigg(\frac{1-\xi}{2}\bigg)^2 \frac{\partial \Psi^0}{\partial \xi}= (1+\xi) (1-\xi) \frac{\partial \Psi^0}{\partial \xi}  ,\\
&=\frac{GM}{a}\bigg[\frac{a(1-x)}{c}\bigg( \frac{r}{c}\bigg)^2 \bigg(\frac{1}{1+(r/c)^2}\bigg)^{3/2} 
+ \frac{x}{q}\bigg(\frac{1+\xi}{2}\bigg) \bigg(\frac{1-\xi}{2}\bigg)^{1/2}  \bigg(1- \varepsilon\bigg(\frac{x}{q}\bigg)^{\Gamma-2}\bigg(\frac{1-\xi}{2}\bigg)^{\frac{3\Gamma}{2}-2} \bigg) \bigg], \\
&=\frac{GMx}{aq}\bigg[\frac{a}{c}\frac{q(1-x)}{x}\bigg( \frac{r}{c}\bigg)^2 \bigg(\frac{1}{1+(r/c)^2}\bigg)^{3/2} 
+ \bigg(\frac{1+\xi}{2}\bigg) \bigg(\frac{1-\xi}{2}\bigg)^{1/2}  \bigg(1- \varepsilon\bigg(\frac{x}{q}\bigg)^{\Gamma-2}\bigg(\frac{1-\xi}{2}\bigg)^{\frac{3\Gamma}{2}-2} \bigg) \bigg],
\end{align*}
with
$$\frac{1-\xi}{2}= \frac{a^2}{r^2+a^2}= \frac{1}{1+(r/a)^2}; \quad \frac{1+\xi}{2}= \frac{r^2}{r^2+a^2}= \frac{(r/a)^2}{1+(r/a)^2}.$$

Note that for $x=1$ (no bulb) and $q=1$ (purely self-gravitating system), we recover the expression of Toomre
\begin{align*}
\vt^0 &=\bigg(\frac{GM}{a} \bigg)^{1/2}\bigg(\frac{1+\xi}{2}\bigg)^{1/2} \bigg(\frac{1-\xi}{2}\bigg)^{1/4}  \sqrt{1- \varepsilon \bigg(\frac{1-\xi}{2}\bigg)^{\frac{3\Gamma}{2}-2}} .
\end{align*}

The perturbative equation at 1st order are
\begin{align*}
&\frac{\partial \vr^p}{\partial t} + \frac{\vt^0}{r} \frac{\partial \vr^p}{\partial \theta} - 2\frac{\vt^0 \vt^p}{r} = -\frac{\partial \Psi^p}{\partial r} ,\\
&\frac{\partial \vt^p}{\partial t} + \vr^p \frac{\partial \vt^0}{\partial r} + \frac{\vt^0}{r} \frac{\partial \vt^p}{\partial \theta} + \frac{\vr^p\vt^0}{r} = -\frac{1}{r}\frac{\partial \Psi^p}{\partial \theta} ,\\
& \Delta \Phisg^p = 4\pi G \frac{1}{q} \Sigmad^p \delta (z), \\
& \psi^p =  \frac{\kappa \Gamma}{q^{\Gamma-1}} (\Sigmad^0)^{\Gamma-2} \Sigmad^p .
\end{align*}

We define $X^p(r,\theta,t) = \sum_{m\in \mathbb{Z}} X_m^p(r,t) e^{i m \theta}$ and look for a temporal dependency in $e^{-i \omega t}$.
Aoki \& Iye say that there is this following correspondance between surface density and gravitational potential through the Poisson equation:
\begin{align}
(\Sigmad)_m^p (r,\theta) &= \frac{xM}{2\pi a^2} \bigg(\frac{1-\xi}{2}\bigg)^{3/2} \sum_{n=|m|}^{\infty} \anm \hPnm(\xi) e^{-i \omega t} ,\\
(\Phisg)_m^p (r,\theta) &= -\frac{GMx}{a q} \bigg(\frac{1-\xi}{2}\bigg)^{1/2} \sum_{n=|m|}^{\infty} \frac{\anm}{2n+1} \hPnm(\xi) e^{-i \omega t} ,\\
(\psi)_m^p (r,\theta) &= \kappa \Gamma \bigg(\frac{M}{2\pi a^2} \bigg)^{\Gamma-1} \bigg(\frac{x}{q}\bigg)^{\Gamma-1} \bigg(\frac{1-\xi}{2}\bigg)^{3/2(\Gamma-1)} \sum_{n=|m|}^{\infty} \anm \hPnm(\xi) e^{-i \omega t} ,\\
(\Psi)_m^p (r,\theta) &= \frac{GMx}{a q } \bigg(\frac{1-\xi}{2}\bigg)^{1/2}\sum_{n=|m|}^{\infty} \bigg[\frac{\varepsilon}{3}  \bigg(\frac{x}{q}\bigg)^{\Gamma-2} \bigg(\frac{1-\xi}{2}\bigg)^{\frac{3\Gamma}{2}-2}  -\frac{1}{2n+1}  \bigg]\anm \hPnm(\xi)e^{-i \omega t} .
\end{align}

Let us decompose the velocity components based on their equilibrium expression:
\begin{align}
(\vr)_m^p &= i \frac{m}{|m|} \bigg(\frac{GMx}{aq}\bigg)^{1/2} \bigg(\frac{1+\xi}{2}\bigg)^{-1/2} \bigg(\frac{1-\xi}{2}\bigg)^{1/4} \sum_{n=|m|}^{\infty} \bnm \hPnm(\xi) e^{-i \omega t} ,\\
(\vt)_m^p &=\bigg(\frac{GMx}{aq}\bigg)^{1/2} \bigg(\frac{1+\xi}{2}\bigg)^{-1/2} \bigg(\frac{1-\xi}{2}\bigg)^{1/4} \sum_{n=|m|}^{\infty} \cnm \hPnm(\xi) e^{-i \omega t} .
\end{align}

Letting $X_m^p = X^1  e^{-i \omega t}$, this yields the set of equations
\begin{align}
&i(-\omega + m \Omega)(\Sigmad)^1 + \frac{1}{r} \frac{\rd (r\Sigma_0 (\vr)^1)}{\rd r} + \frac{im\Sigma^0 (\vt)^1}{r} = 0,\\
&\frac{\rd (\Psi)^1}{\rd r} + i(-\omega + m \Omega)(\vr)^1 - 2 \Omega (\vt)^1 = 0,\\
&i m \frac{(\Psi)^1}{r} + \frac{\alpha^2}{2\Omega} (\vr)^1 + i(-\omega + m \Omega)(\vt)^1 = 0,
\end{align}
where $\Omega = \vt^0/r$ is the angular velocity and $\alpha^2=4\Omega^2[1+r/(2\Omega)\cdot(\rd \Omega /\rd r)]$ is the epicyclic frequency. Using the relation 
$$\int_{-1}^{1} \rd \xi \hPnm(\xi)\hPlm(\xi) = \delta_{nl},$$
and defining $\Omegaref=(GMx/(a^3 q))^{1/2}$, $\Sigmaref=xM/(2\pi a^2)$   such that $\homega = \omega/\Omegaref$, $\hOmega = \Omega/\Omegaref$, $\halpha = \alpha/\Omegaref$, $\hSigma=\Sigma/\Sigmaref$ and $\lambda=\frac{|m|}{m}\homega$, we obtain the matrix equations
\begin{align*}
&\sum_{n=|m|}^{\infty}  A_{ln} \anm  +\sum_{n=|m|}^{\infty}  B_{ln}\bnm+\sum_{n=|m|}^{\infty}  C_{ln}\cnm= \lambda \alm ,\\
&\sum_{n=|m|}^{\infty}D_{ln}\anm + \sum_{n=|m|}^{\infty}   A_{ln} \bnm+  \sum_{n=|m|}^{\infty} F_{ln} \cnm  = \lambda   \blm ,\\
&\sum_{n=|m|}^{\infty}G_{ln}\anm + \sum_{n=|m|}^{\infty}   H_{ln} \bnm+  \sum_{n=|m|}^{\infty} A_{ln} \cnm  = \lambda   \blm ,
 \end{align*}
where we defined
\begin{align*}
A_{ln} &= |m| \int_{-1}^{1} \rd \xi  \hPlm(\xi)\hOmega(\xi)\hPnm(\xi) , \\
B_{ln} &= 4 \int_{-1}^{1} \rd \xi  \hPlm(\xi) \bigg(\frac{1-\xi}{2}\bigg)^{1/2} \frac{\rd}{\rd \xi} \bigg[\bigg(\frac{1-\xi}{2}\bigg)^{5/4}\hPnm(\xi)\bigg], \\
C_{ln} &= |m| \int_{-1}^{1} \rd \xi \hPlm(\xi) \bigg(\frac{1-\xi}{2}\bigg)^{3/4}\bigg(\frac{1+\xi}{2}\bigg)^{-1}   \hPnm(\xi), \\
D_{ln} &= 4 \int_{-1}^{1} \rd \xi \hPlm(\xi)  \bigg(\frac{1-\xi}{2}\bigg)^{5/4} \bigg(\frac{1+\xi}{2}\bigg) 
\frac{\rd}{\rd \xi} \bigg[\bigg(\frac{1}{2n+1}-\frac{\varepsilon}{3}  \bigg(\frac{x}{q}\bigg)^{\Gamma-2} \bigg(\frac{1-\xi}{2}\bigg)^{\frac{3\Gamma}{2}-2}\bigg) \bigg(\frac{1-\xi}{2}\bigg)^{1/2}\hPnm(\xi) \bigg] ,\\
F_{ln} &=2 \int_{-1}^{1} \rd \xi  \hPlm(\xi)\hOmega(\xi)\hPnm(\xi) , \\
G_{ln} &= -|m| \int_{-1}^{1} \rd \xi \hPlm(\xi)  \bigg(\frac{1-\xi}{2}\bigg)^{3/4} 
\bigg(\frac{1}{2n+1}-\frac{\varepsilon}{3}  \bigg(\frac{x}{q}\bigg)^{\Gamma-2} \bigg(\frac{1-\xi}{2}\bigg)^{\frac{3\Gamma}{2}-2}\bigg) \hPnm(\xi)  ,\\
H_{ln} &=  \int_{-1}^{1} \rd \xi  \hPlm(\xi)\frac{\halpha^2(\xi)}{2\hOmega(\xi)}\hPnm(\xi) .
\end{align*}
with
\begin{align*}
\hOmega(\xi) &=\sqrt{\frac{1-\xi}{1+\xi}} \sqrt{\frac{a}{c}\frac{q(1-x)}{x}\bigg( \frac{r}{c}\bigg)^2 \bigg(\frac{1}{1+(r/c)^2}\bigg)^{3/2} 
+ \bigg(\frac{1+\xi}{2}\bigg) \bigg(\frac{1-\xi}{2}\bigg)^{1/2}  \bigg[1- \varepsilon\bigg(\frac{x}{q}\bigg)^{\Gamma-2}\bigg(\frac{1-\xi}{2}\bigg)^{\frac{3\Gamma}{2}-2} \bigg] } ,\\
\frac{\halpha^2(\xi)}{2\hOmega(\xi)} &= 2\hOmega(\xi) \bigg[1+\frac{r}{2\hOmega}\frac{\rd \hOmega }{\rd r}\bigg] = 2\hOmega(\xi) \bigg[1+\frac{(1+\xi)(1-\xi)}{2\hOmega}\frac{\rd \hOmega }{\rd \xi} \bigg],
\end{align*}
where
$$\frac{\rd(r/a)}{\rd \xi} = \frac{\rd}{\rd \xi} \bigg[\sqrt{\frac{1+\xi}{1-\xi}}\bigg] = (1-\xi)^{-3/2}(1+\xi)^{-1/2}$$

Since $(r/c)^2 = (a/c)^2 \cdot (r/a)^2 =  (a/c)^2 \cdot (1+\xi)/(1-\xi)$, we have that
\begin{align*}
\hOmega(\xi) &=\bigg(\frac{1-\xi}{2}\bigg)^{3/4}  \sqrt{\bigg( \frac{a}{c}\bigg)^3\frac{q(1-x)}{x} \bigg(\frac{1-\xi}{2}\bigg)^{-3/2} \bigg(\frac{1}{1+(r/c)^2}\bigg)^{3/2} 
+   \bigg[1- \varepsilon\bigg(\frac{x}{q}\bigg)^{\Gamma-2}\bigg(\frac{1-\xi}{2}\bigg)^{\frac{3\Gamma}{2}-2} \bigg] } .
\end{align*}

\section{Case $\Gamma=4/3$}

The expressions become
\begin{align*}
\hOmega(\xi) &=\bigg(\frac{1-\xi}{2}\bigg)^{3/4}  \sqrt{\bigg( \frac{a}{c}\bigg)^3\frac{q(1-x)}{x} \bigg(\frac{1-\xi}{2}\bigg)^{-3/2} \bigg(\frac{1}{1+(r/c)^2}\bigg)^{3/2} 
+   \bigg[1- \varepsilon\bigg(\frac{q}{x}\bigg)^{\frac{2}{3}} \bigg]  } , \\
\frac{\halpha^2(\xi)}{2\hOmega(\xi)} & = 2\hOmega(\xi) \bigg[1+\frac{(1+\xi)(1-\xi)}{2\hOmega}\frac{\rd \hOmega }{\rd \xi} \bigg],
\end{align*}
with the matrix elements
\begin{align*}
A_{ln} &= |m| \int_{-1}^{1} \rd \xi  \hPlm(\xi)\hOmega(\xi)\hPnm(\xi) , \\
B_{ln} &= 4 \int_{-1}^{1} \rd \xi  \hPlm(\xi) \bigg(\frac{1-\xi}{2}\bigg)^{1/2} \frac{\rd}{\rd \xi} \bigg[\bigg(\frac{1-\xi}{2}\bigg)^{5/4}\hPnm(\xi)\bigg], \\
C_{ln} &= |m| \int_{-1}^{1} \rd \xi \hPlm(\xi) \bigg(\frac{1-\xi}{2}\bigg)^{3/4}\bigg(\frac{1+\xi}{2}\bigg)^{-1}   \hPnm(\xi), \\
D_{ln} &= 4 \bigg(\frac{1}{2n+1}-\frac{\varepsilon}{3}  \bigg(\frac{q}{x}\bigg)^{\frac{2}{3}} \bigg) \int_{-1}^{1} \rd \xi \hPlm(\xi)  \bigg(\frac{1-\xi}{2}\bigg)^{5/4} \bigg(\frac{1+\xi}{2}\bigg) 
\frac{\rd}{\rd \xi} \bigg[ \bigg(\frac{1-\xi}{2}\bigg)^{1/2}\hPnm(\xi) \bigg] ,\\
F_{ln} &=2 \int_{-1}^{1} \rd \xi  \hPlm(\xi)\hOmega(\xi)\hPnm(\xi) , \\
G_{ln} &= -|m| \bigg(\frac{1}{2n+1}-\frac{\varepsilon}{3}  \bigg(\frac{q}{x}\bigg)^{\frac{2}{3}} \bigg)  \int_{-1}^{1} \rd \xi \hPlm(\xi)  \bigg(\frac{1-\xi}{2}\bigg)^{3/4} 
\hPnm(\xi)  ,\\
H_{ln} &=  \int_{-1}^{1} \rd \xi  \hPlm(\xi)\frac{\halpha^2(\xi)}{2\hOmega(\xi)}\hPnm(\xi) .
\end{align*}

Integrals $A_{ln}$ and  $F_{ln}$ are proportional. With the addition of $H_{ln}$, those 3 integrals must be computed numerically because of the non-trivial shift in their expression induced by the bulb potential. As for $B_{ln}$, $C_{ln}$, $D_{ln}$ and $G_{ln}$, their can be can expressed in terms of the two following integrals 
\begin{align}
\hI(l,n) &= \int_{-1}^{1} \rd \xi  \bigg(\frac{1-\xi}{2}\bigg)^{3/4} \hPlm(\xi)  \hPnm(\xi), \\
\hJ(l,n) &= \int_{-1}^{1} \rd \xi  \bigg(\frac{1-\xi}{2}\bigg)^{3/4}  \bigg(\frac{1+\xi}{2}\bigg)^{-1}\hPlm(\xi)  \hPnm(\xi) ,
\end{align}
as
\begin{align*}
B_{ln} &= \frac{1}{2} \bigg[ \sqrt{\frac{(2l+1)(l+m+1)(l-m+1)}{2l+3}} \hJ(l+1,n)+ \hJ(l,n)\\
&\quad \quad  -\sqrt{\frac{(2l+1)(l+m)(l-m)}{2l-1}} \hJ(l-1,n)\bigg] ,\\
C_{ln} &= m \hJ(l,n) ,\\
D_{ln} &= \frac{1}{2}  \bigg(\frac{1}{2n+1}-\frac{\varepsilon}{3}  \bigg(\frac{q}{x}\bigg)^{\frac{2}{3}} \bigg)
\bigg[ -\sqrt{\frac{(2n+1)(n+m+1)(n-m+1)}{2n+3}} \hI(l,n+1) \\
&\quad \quad \quad \quad \quad \quad\quad \quad\quad \quad \quad \quad- \hI(l,n) \\
&\quad \quad \quad \quad \quad \quad\quad \quad\quad \quad \quad \quad  +\sqrt{\frac{(2n+1)(n+m)(n-m)}{2n-1}} \hI(l,n-1)\bigg] ,\\
G_{ln} &= -|m| \bigg(\frac{1}{2n+1}-\frac{\varepsilon}{3}  \bigg(\frac{q}{x}\bigg)^{\frac{2}{3}} \bigg)  \hI(l,n) .
\end{align*}
where $\hI(l,n)$ and $\hJ(l,n)$ can be computed by recursion and using the symmetry $l\leftrightarrow n$. Defining
\begin{align}
\hIp(l,n) &= \int_{-1}^{1} \rd \xi \, \xi \bigg(\frac{1-\xi}{2}\bigg)^{3/4} \hPlm(\xi)  \hPnm(\xi), \\
\hJp(l,n) &= \int_{-1}^{1} \rd \xi \, \xi \bigg(\frac{1-\xi}{2}\bigg)^{3/4}  \bigg(\frac{1+\xi}{2}\bigg)^{-1}\hPlm(\xi)  \hPnm(\xi) ,
\end{align}
Starting from(Aoki79, A16)
\begin{align*}
\hI(l,n) &= \sqrt{\frac{(2n+1)(2n-1)}{(n+m)(n-m)}} \hIp(l,n-1)-\sqrt{\frac{(n+m-1)(n-m-1)(2n+1)}{(n+m)(n-m)(2n-3)}} \hI(l,n-2) ,\\
\hIp(l,n-1) &= \sqrt{\frac{(l+m+1)(l-m+1)}{(2l+1)(2l+3)}} \hI(l+1,n-1)+\sqrt{\frac{(l+m)(l-m)}{(2l+1)(2l-1)}} \hI(l-1,n-1) ,
\end{align*}
hence to compute until $l,n=m+N$, we need to compute $\hI(l',m)$ until $l'=m+2N$. By convention (for the recursion), we have set $\hI(l',n')=0$ for $l'<m$ or  $n'<m$.
We initialize with
\begin{align*}
\hI(l,m) &= \frac{l-3/4-m-1}{l+3/4+m+1} \sqrt{\frac{(l+m)(2l+1)}{(l-m)(2l-1)}} \hI(l-1,m) ,\\
\hI(m,m) &= 2^m \prod_{k=0}^{m} \frac{2k+1}{3/4 + m + 1 + k} .
\end{align*}

We proceed as follows:

- Compute the line $n=m$: $\hI(m,m)$, $\hI(m+1,m)$ , ... , $\hI(m+2N,m)$

- Complete the line $l=m$ by symmetry

- Compute the line $n=m+1$: $\hI(m+1,m+1)$, $\hI(m+2,m+1)$ , ... , $\hI(m+2N-1,m+1)$

- Complete the line $l=m+1$ by symmetry

- Compute the line $n=m+2$: $\hI(m+2,m+2)$, $\hI(m+3,m+2)$ , ... , $\hI(m+2N-2,m+2)$

- ...

- Compute the line $n=m+N-1$: $\hI(m+N-1,m+N-1)$, $\hI(m+N+1,m+N-1)$.

- Complete the line $l=m+N-1$ by symmetry

- Compute the line $n=m+N$: $\hI(m+N,m+N)$.

As for $\hJ$, let
\begin{align}
\hI_{\alpha}(l,m) &= \frac{(-1)^{l-m}(2m-1)!!2^{m+1}\Gamma(\alpha+1)\Gamma(\alpha+m+1)(l+m)!}{\Gamma(\alpha+1-l+m)\Gamma(\alpha+m+l+2)(l-m)!} 
\end{align}
with $\hI_{3/4}(l,m) =\hI(l,m)$. We can also compute it by recursion using the formulae
\begin{align*}
\hJ(l,n) &= \sqrt{\frac{(2n+1)(2n-1)}{(n+m)(n-m)}} \hJp(l,n-1)-\sqrt{\frac{(n+m-1)(n-m-1)(2n+1)}{(n+m)(n-m)(2n-3)}} \hJ(l,n-2) ,\\
\hJp(l,n-1) &= \sqrt{\frac{(l+m+1)(l-m+1)}{(2l+1)(2l+3)}} \hJ(l+1,n-1)+\sqrt{\frac{(l+m)(l-m)}{(2l+1)(2l-1)}} \hJ(l-1,n-1) ,
\end{align*}
hence to compute until $l,n=m+N$, we need to compute $\hI(l',m)$ until $l'=m+2N$. By convention (for the recursion), we have set $\hI(l',n')=0$ for $l'<m$ or  $n'<m$.
We initialize with
\begin{align*}
\hJ(l,m) &=\sqrt{\frac{(l-m)(l-m-1)(2l+1)}{(l+m)(l+m-1)(2l-3)}} \hJ(l-2,m)+ 4 \sqrt{\frac{(2m+1)(2l+1)(2l-1)}{2m(l+m)(l+m-1)}} \hI_{7/3}(l-1,m-1) ,\\
\hJ(m,m) &= \frac{2^m}{m} \prod_{k=1}^{m} \frac{2k+1}{3/4 + m + k} ,\\
\hJ(m+1,m) &= -\frac{7}{4}\frac{2^m\sqrt{2m+3}}{m} \prod_{k=0}^{m} \frac{2k+1}{3/4 + m + 1 +k} = -\frac{7\sqrt{2m+3}}{4(3/4+2m+1)} \hJ(m,m).
\end{align*}
We can compute the $\hI_{7/4}$ part by recursion. Indeed,
\begin{align}
\hI_{\alpha}(l,m) &= \frac{l-\alpha-m-1}{l+\alpha+m+1} \sqrt{\frac{(l+m)(2l+1)}{(l-m)(2l-1)}} \hI_{\alpha}(l-1,m) ,\\
\hI_{\alpha}(m,m) &= 2^m \prod_{k=0}^{m} \frac{2k+1}{\alpha + m + 1 + k} .
\end{align}

Hence, we need to compute beforehand the values $\hI_{\alpha}(l',m-1)$ for $l=m-1,...,m+2N-1$, and then apply the same process as for $\hI$. 

As for the numerical integral, we use a simple midpoint rule with $K$ points. Those integrals have the form
\begin{align*}
I_{ln} &= \int_{-1}^{1} \rd \xi \hPlm(\xi) \phi(\xi) \hPnm(\xi) \approx \frac{2}{K} \sum_{k=1}^{K} \hPlm(\xi_{k}) \phi(\xi_{k}) \hPnm(\xi_{k}),
\end{align*}
where $\xi_{k} = -1 + (2/K)(k-1/2)$. As we with to compute those elements for $m \leq l,n \leq m+N$, we have to compute the $\hPlm(\xi_{k})$ for $n=m,...,m+N$ and $k=1,...,K$. To that end, We compute before hand a table of the values $\{\hPlm(\xi_{k})\}_{(n,k)}$ and of the values $\{\phi(\xi_{k})\}_{k}$. The Legendre associated functions can be efficiently computed by using the Julia library "SphericalHarmomics", in which we use the function "computePlmcostheta($\theta,l_{\max},m$)" which compute $\hPlm(\cos(\theta))/\sqrt{\pi}$ for all $n=0,...,l_{\max}$ at a given $m$.

\section{Eigenmodes}

\subsection{Physical eigenvalues}

Instead of $q=M_{\mathrm{disk}}/M$, we may use $\alpha = M_{\mathrm{DH}}/M_{\mathrm{disk}}$ such that $q=1/(1+\alpha)$. The self-gravitating case corresponds to $q=1$ and $\alpha=0$.

\subsection{Fastest eigenmode}
Consider the mode $m$ with fastest physical eigenvalue $\omega$ at truncation $m$. Then associated surface density is given by
\begin{align}
(\widehat{\Sigmad})_m^p (r,\theta) &= x \bigg(\frac{1-\xi}{2}\bigg)^{3/2} \sum_{n=|m|}^{\infty} \anm \hPnm(\xi) e^{i m \theta}.
\end{align}

Truncating at $m+N$, we recover the approximation 
\begin{align}
(\widehat{\Sigmad})_m^p (r,\theta) &\simeq x \bigg(\frac{1-\xi}{2}\bigg)^{3/2} \sum_{n=|m|}^{|m|+N}  \hPnm(\xi)\Re (\anm e^{i m \theta}).
\end{align}






\section{Self-gravitating thin disk with constant total mass}
\label{sec:sg_disk_cst_mass}

Consider a self gravitating disk with surface density $\Sigma$ and gravitational potential $\Phi$. Let $\bv$ be its velocity field and $P$ its pression field. Then it is described by the system
\begin{align}
&\frac{\partial \Sigma}{\partial t} + \bnab \cdot (\Sigma \bv) = 0 ,\\
&\frac{\partial \bv}{\partial t} + (\bv \cdot \bnab)(\bv) = -\frac{1}{\Sigma} \bnab P - \bnab \Phi ,\\
& \Delta \Phi = 4\pi G \Sigma \delta (z).
\end{align}
Here, $\Phi =\Phib+  \Phid +  \Phidh=\Phib +  \Phisg$. Furthermore, $\Phib$  is such that it doesn't yield any contribution to the surface density, i.e. $\Delta \Phib=0$, nor the pressure field, hence $\Sigma =\Sigmad+\Sigmadh$ and $P =P_{\rm{disk}}+P_{\rm{DH}}$. Letting $q=\Sigma/\Sigmad$, we have

\begin{equation}
\Sigma=\Sigmad+\Sigma_{\rm{DH}}=\Sigmad\bigg(1+\frac{\Sigmadh}{\Sigmad}\bigg) =q \Sigmad.
\end{equation}

Suppose that we have a polytrope gas such that $P=\kappa \Sigma^{\Gamma}$ and let
\begin{equation}
\psi = \int \frac{\rd P(\Sigma)}{\Sigma} \Leftrightarrow \bnab \psi = \frac{\bnab P}{\Sigma}.
\end{equation}
 Then
 \begin{equation}
\psi = \frac{\kappa \Gamma}{\Gamma-1} \Sigma^{\Gamma-1}.
\end{equation}
 
 Letting $\Psi = \Phi + \psi$, we obtain
\begin{equation}
\frac{\partial \bv}{\partial t} + (\bv \cdot \bnab)(\bv) = - \bnab \Psi.
\end{equation}

If $\Sigma_{\rm{disk}}\propto\Sigma_{\rm{DH}}$ then $q\in [1,+\infty[$ is constant and $q=1+\Mdh/\Md$. Let  $M$ be the total mass of the galactic disk+bulb+DH. Let $x=\Md/M$.  Then

\begin{align*}
\Md &= x M,\\
\Mdh &= (q-1) x M,\\
\Mb &= (1 - qx)M  .
\end{align*}
where $qx=(1+\Mdh/\Md)(\Md/M)=(\Md+\Mdh)/M < 1$. Therefore from the mass fractions we can obtain $x$ and $q$ by taking
\begin{align*}
x&=  \frac{\Md }{M},\\
q &= 1+\frac{\Mdh/M}{\Md/M}  .
\end{align*}

The system of equation can be rewritten as 
\begin{align}
&\frac{\partial \Sigmad}{\partial t} + \frac{1}{r} \frac{\partial(r\Sigmad \vr)}{\partial r} + \frac{1}{r} \frac{\partial(\Sigmad \vt)}{\partial \theta} = 0 ,\\
&\frac{\partial \vr}{\partial t} + \vr \frac{\partial \vr}{\partial r} + \frac{\vt}{r} \frac{\partial \vr}{\partial \theta} - \frac{(\vt)^2}{r} = -\frac{\partial \Psi}{\partial r} ,\\
&\frac{\partial \vt}{\partial t} + \vr \frac{\partial \vt}{\partial r} + \frac{\vt}{r} \frac{\partial \vt}{\partial \theta} + \frac{\vr\vt}{r} = -\frac{1}{r}\frac{\partial \Psi}{\partial \theta} ,\\
& \Delta \Phisg = 4\pi G q\Sigmad \delta (z).
\end{align}

 Let $\Phib = - G M (1-q x)/\sqrt{c^2+r^2}$ with $x\in [0,1]$. An equilibrium state is given by the Plummer equilibrium, such that,  letting $\xi = (r^2-a^2)/(r^2+a^2) $, that is, $r/a=\sqrt{(1+\xi)/(1-\xi)}$,
\begin{align*}
&\vr^0 = 0,\quad \quad (\vt^0)^2 = r \frac{\partial \Psi^0}{\partial r}, \\
& \psi^0 = \frac{\kappa \Gamma}{\Gamma-1} (\Sigma^0)^{\Gamma-1} = \frac{\kappa \Gamma q^{\Gamma-1}}{(\Gamma-1)} (\Sigmad^0)^{\Gamma-1},\\
& \Sigmad^0 = \frac{xM}{2\pi a^2} \frac{1}{(1+(r/a)^2)^{3/2}} =\frac{xM}{2\pi a^2}   \bigg(\frac{1-\xi}{2}\bigg)^{3/2},\\
& \Phisg^0 = -\frac{G Mqx}{a } \frac{1}{\sqrt{1+(r/a)^2}} = -\frac{G M qx}{a } \bigg(\frac{1-\xi}{2}\bigg)^{1/2} .
\end{align*}


Therefore, letting  $$\varepsilon = \frac{U(x=1,q=1)}{|E(x=1,q=1)|} =\frac{3a \kappa \Gamma}{GM} \bigg(\frac{M}{2\pi a^2}\bigg)^{\Gamma-1},$$  we obtain
\begin{align*}
\Psi^0 &=  - \frac{G M (1-qx)}{\sqrt{c^2+r^2}}-\frac{G M qx}{a }\bigg(\frac{1-\xi}{2}\bigg)^{1/2} + \frac{\kappa \Gamma}{(\Gamma-1)} (q\Sigmad^0)^{\Gamma-1} ,\\
{}&=  -\frac{G M (1-qx)}{\sqrt{c^2+r^2}}+\frac{G M }{a} \bigg[-(qx)\bigg(\frac{1-\xi}{2}\bigg)^{1/2} + \frac{\varepsilon(qx)^{\Gamma-1}}{3(\Gamma-1)}\bigg(\frac{1-\xi}{2}\bigg)^{3(\Gamma-1)/2} \bigg] ,\\
&=  \frac{G Mqx}{a}\bigg[-\frac{a}{c}\frac{ 1-qx}{qx}\frac{1}{ \sqrt{1+(r/c)^2}} -\bigg(\frac{1-\xi}{2}\bigg)^{1/2} + \frac{\varepsilon(xq)^{\Gamma-2}}{3(\Gamma-1)}\bigg(\frac{1-\xi}{2}\bigg)^{3(\Gamma-1)/2} \bigg],\\
(\vt^0)^2 &= r \frac{\partial \Psi^0}{\partial r} =r \frac{\partial \Psi^0}{\partial \xi} \frac{\partial \xi}
{\partial r} = 4\bigg(\frac{r}{a}\bigg)^2    \bigg(\frac{1-\xi}{2}\bigg)^2 \frac{\partial \Psi^0}{\partial \xi}= (1+\xi) (1-\xi) \frac{\partial \Psi^0}{\partial \xi}  ,\\
&=\frac{GM}{a}\bigg[\frac{a(1-qx)}{c}\bigg( \frac{r}{c}\bigg)^2 \bigg(\frac{1}{1+(r/c)^2}\bigg)^{3/2} 
+ qx\bigg(\frac{1+\xi}{2}\bigg) \bigg(\frac{1-\xi}{2}\bigg)^{1/2}  \bigg(1- \varepsilon(qx)^{\Gamma-2}\bigg(\frac{1-\xi}{2}\bigg)^{\frac{3\Gamma}{2}-2} \bigg) \bigg], \\
&=\frac{GMqx}{a}\bigg[\frac{a}{c}\frac{1-qx}{qx}\bigg( \frac{r}{c}\bigg)^2 \bigg(\frac{1}{1+(r/c)^2}\bigg)^{3/2} 
+ \bigg(\frac{1+\xi}{2}\bigg) \bigg(\frac{1-\xi}{2}\bigg)^{1/2}  \bigg(1- \varepsilon(qx)^{\Gamma-2}\bigg(\frac{1-\xi}{2}\bigg)^{\frac{3\Gamma}{2}-2} \bigg) \bigg],
\end{align*}
with
$$\frac{1-\xi}{2}= \frac{a^2}{r^2+a^2}= \frac{1}{1+(r/a)^2}; \quad \frac{1+\xi}{2}= \frac{r^2}{r^2+a^2}= \frac{(r/a)^2}{1+(r/a)^2}.$$

Note that for $x=1$ (only disk) and $q=1$ (purely self-gravitating system), we recover the expression of Toomre
\begin{align*}
\vt^0 &=\bigg(\frac{GM}{a} \bigg)^{1/2}\bigg(\frac{1+\xi}{2}\bigg)^{1/2} \bigg(\frac{1-\xi}{2}\bigg)^{1/4}  \sqrt{1- \varepsilon \bigg(\frac{1-\xi}{2}\bigg)^{\frac{3\Gamma}{2}-2}} .
\end{align*}

The perturbative equation at 1st order are
\begin{align*}
&\frac{\partial \vr^p}{\partial t} + \frac{\vt^0}{r} \frac{\partial \vr^p}{\partial \theta} - 2\frac{\vt^0 \vt^p}{r} = -\frac{\partial \Psi^p}{\partial r} ,\\
&\frac{\partial \vt^p}{\partial t} + \vr^p \frac{\partial \vt^0}{\partial r} + \frac{\vt^0}{r} \frac{\partial \vt^p}{\partial \theta} + \frac{\vr^p\vt^0}{r} = -\frac{1}{r}\frac{\partial \Psi^p}{\partial \theta} ,\\
& \Delta \Phisg^p = 4\pi G q \Sigmad^p \delta (z), \\
& \psi^p =  \kappa \Gamma q^{\Gamma-1} (\Sigmad^0)^{\Gamma-2} \Sigmad^p .
\end{align*}

We define $X^p(r,\theta,t) = \sum_{m\in \mathbb{Z}} X_m^p(r,t) e^{i m \theta}$ and look for a temporal dependency in $e^{-i \omega t}$.
Aoki \& Iye say that there is this following correspondance between surface density and gravitational potential through the Poisson equation:
\begin{align}
(\Sigmad)_m^p (r,\theta) &= \frac{xM}{2\pi a^2} \bigg(\frac{1-\xi}{2}\bigg)^{3/2} \sum_{n=|m|}^{\infty} \anm \hPnm(\xi) e^{-i \omega t} ,\\
(\Phisg)_m^p (r,\theta) &= -\frac{GMqx}{a} \bigg(\frac{1-\xi}{2}\bigg)^{1/2} \sum_{n=|m|}^{\infty} \frac{\anm}{2n+1} \hPnm(\xi) e^{-i \omega t} ,\\
(\psi)_m^p (r,\theta) &= \kappa \Gamma \bigg(\frac{M}{2\pi a^2} \bigg)^{\Gamma-1} (xq)^{\Gamma-1} \bigg(\frac{1-\xi}{2}\bigg)^{3/2(\Gamma-1)} \sum_{n=|m|}^{\infty} \anm \hPnm(\xi) e^{-i \omega t} ,\\
(\Psi)_m^p (r,\theta) &= \frac{GMqx}{a } \bigg(\frac{1-\xi}{2}\bigg)^{1/2}\sum_{n=|m|}^{\infty} \bigg[\frac{\varepsilon}{3} (qx)^{\Gamma-2} \bigg(\frac{1-\xi}{2}\bigg)^{\frac{3\Gamma}{2}-2}  -\frac{1}{2n+1}  \bigg]\anm \hPnm(\xi)e^{-i \omega t} .
\end{align}

Let us decompose the velocity components based on their equilibrium expression:
\begin{align}
(\vr)_m^p &= i \frac{m}{|m|} \bigg(\frac{GMqx}{a}\bigg)^{1/2} \bigg(\frac{1+\xi}{2}\bigg)^{-1/2} \bigg(\frac{1-\xi}{2}\bigg)^{1/4} \sum_{n=|m|}^{\infty} \bnm \hPnm(\xi) e^{-i \omega t} ,\\
(\vt)_m^p &=\bigg(\frac{GMqx}{a}\bigg)^{1/2} \bigg(\frac{1+\xi}{2}\bigg)^{-1/2} \bigg(\frac{1-\xi}{2}\bigg)^{1/4} \sum_{n=|m|}^{\infty} \cnm \hPnm(\xi) e^{-i \omega t} .
\end{align}

Letting $X_m^p = X^1  e^{-i \omega t}$, this yields the set of equations
\begin{align}
&i(-\omega + m \Omega)(\Sigmad)^1 + \frac{1}{r} \frac{\rd (r\Sigma_0 (\vr)^1)}{\rd r} + \frac{im\Sigma^0 (\vt)^1}{r} = 0,\\
&\frac{\rd (\Psi)^1}{\rd r} + i(-\omega + m \Omega)(\vr)^1 - 2 \Omega (\vt)^1 = 0,\\
&i m \frac{(\Psi)^1}{r} + \frac{\alpha^2}{2\Omega} (\vr)^1 + i(-\omega + m \Omega)(\vt)^1 = 0,
\end{align}
where $\Omega = \vt^0/r$ is the angular velocity and $\alpha^2=4\Omega^2[1+r/(2\Omega)\cdot(\rd \Omega /\rd r)]$ is the epicyclic frequency. Using the relation 
$$\int_{-1}^{1} \rd \xi \hPnm(\xi)\hPlm(\xi) = \delta_{nl},$$
and defining $\Omegaref=(GMqx/(a^3 ))^{1/2}$, $\Sigmaref=xM/(2\pi a^2)$   such that $\homega = \omega/\Omegaref$, $\hOmega = \Omega/\Omegaref$, $\halpha = \alpha/\Omegaref$, $\hSigma=\Sigma/\Sigmaref$ and $\lambda=\frac{|m|}{m}\homega$, we obtain the matrix equations
\begin{align*}
&\sum_{n=|m|}^{\infty}  A_{ln} \anm  +\sum_{n=|m|}^{\infty}  B_{ln}\bnm+\sum_{n=|m|}^{\infty}  C_{ln}\cnm= \lambda \alm ,\\
&\sum_{n=|m|}^{\infty}D_{ln}\anm + \sum_{n=|m|}^{\infty}   A_{ln} \bnm+  \sum_{n=|m|}^{\infty} F_{ln} \cnm  = \lambda   \blm ,\\
&\sum_{n=|m|}^{\infty}G_{ln}\anm + \sum_{n=|m|}^{\infty}   H_{ln} \bnm+  \sum_{n=|m|}^{\infty} A_{ln} \cnm  = \lambda   \blm ,
 \end{align*}
where we defined
\begin{align*}
A_{ln} &= |m| \int_{-1}^{1} \rd \xi  \hPlm(\xi)\hOmega(\xi)\hPnm(\xi) , \\
B_{ln} &= 4 \int_{-1}^{1} \rd \xi  \hPlm(\xi) \bigg(\frac{1-\xi}{2}\bigg)^{1/2} \frac{\rd}{\rd \xi} \bigg[\bigg(\frac{1-\xi}{2}\bigg)^{5/4}\hPnm(\xi)\bigg], \\
C_{ln} &= |m| \int_{-1}^{1} \rd \xi \hPlm(\xi) \bigg(\frac{1-\xi}{2}\bigg)^{3/4}\bigg(\frac{1+\xi}{2}\bigg)^{-1}   \hPnm(\xi), \\
D_{ln} &= 4 \int_{-1}^{1} \rd \xi \hPlm(\xi)  \bigg(\frac{1-\xi}{2}\bigg)^{5/4} \bigg(\frac{1+\xi}{2}\bigg) 
\frac{\rd}{\rd \xi} \bigg[\bigg(\frac{1}{2n+1}-\frac{\varepsilon}{3}  (qx)^{\Gamma-2} \bigg(\frac{1-\xi}{2}\bigg)^{\frac{3\Gamma}{2}-2}\bigg) \bigg(\frac{1-\xi}{2}\bigg)^{1/2}\hPnm(\xi) \bigg] ,\\
F_{ln} &=2 \int_{-1}^{1} \rd \xi  \hPlm(\xi)\hOmega(\xi)\hPnm(\xi) , \\
G_{ln} &= -|m| \int_{-1}^{1} \rd \xi \hPlm(\xi)  \bigg(\frac{1-\xi}{2}\bigg)^{3/4} 
\bigg(\frac{1}{2n+1}-\frac{\varepsilon}{3}  (qx)^{\Gamma-2} \bigg(\frac{1-\xi}{2}\bigg)^{\frac{3\Gamma}{2}-2}\bigg) \hPnm(\xi)  ,\\
H_{ln} &=  \int_{-1}^{1} \rd \xi  \hPlm(\xi)\frac{\halpha^2(\xi)}{2\hOmega(\xi)}\hPnm(\xi) .
\end{align*}
with
\begin{align*}
\hOmega(\xi) &=\sqrt{\frac{1-\xi}{1+\xi}} \sqrt{\frac{a}{c}\frac{1-qx}{qx}\bigg( \frac{r}{c}\bigg)^2 \bigg(\frac{1}{1+(r/c)^2}\bigg)^{3/2} 
+ \bigg(\frac{1+\xi}{2}\bigg) \bigg(\frac{1-\xi}{2}\bigg)^{1/2}  \bigg[1- \varepsilon(qx)^{\Gamma-2}\bigg(\frac{1-\xi}{2}\bigg)^{\frac{3\Gamma}{2}-2} \bigg] } ,\\
\frac{\halpha^2(\xi)}{2\hOmega(\xi)} &= 2\hOmega(\xi) \bigg[1+\frac{r}{2\hOmega}\frac{\rd \hOmega }{\rd r}\bigg] = 2\hOmega(\xi) \bigg[1+\frac{(1+\xi)(1-\xi)}{2\hOmega}\frac{\rd \hOmega }{\rd \xi} \bigg],
\end{align*}
where
$$\frac{\rd(r/a)}{\rd \xi} = \frac{\rd}{\rd \xi} \bigg[\sqrt{\frac{1+\xi}{1-\xi}}\bigg] = (1-\xi)^{-3/2}(1+\xi)^{-1/2}$$

Since $(r/c)^2 = (a/c)^2 \cdot (r/a)^2 =  (a/c)^2 \cdot (1+\xi)/(1-\xi)$, we have that
\begin{align*}
\hOmega(\xi) &=\bigg(\frac{1-\xi}{2}\bigg)^{3/4}  \sqrt{\bigg( \frac{a}{c}\bigg)^3\frac{1-qx}{qx} \bigg(\frac{1-\xi}{2}\bigg)^{-3/2} \bigg(\frac{1}{1+(r/c)^2}\bigg)^{3/2} 
+   \bigg[1- \varepsilon(qx)^{\Gamma-2}\bigg(\frac{1-\xi}{2}\bigg)^{\frac{3\Gamma}{2}-2} \bigg] } .
\end{align*}

\section{Case $\Gamma=4/3$}

The expressions become
\begin{align*}
\hOmega(\xi) &=\bigg(\frac{1-\xi}{2}\bigg)^{3/4}  \sqrt{\bigg( \frac{a}{c}\bigg)^3\frac{(1-qx)}{qx} \bigg(\frac{1-\xi}{2}\bigg)^{-3/2} \bigg(\frac{1}{1+(r/c)^2}\bigg)^{3/2} 
+   \bigg[1- \varepsilon\bigg(\frac{1}{qx}\bigg)^{\frac{2}{3}} \bigg]  } , \\
\frac{\halpha^2(\xi)}{2\hOmega(\xi)} & = 2\hOmega(\xi) \bigg[1+\frac{(1+\xi)(1-\xi)}{2\hOmega}\frac{\rd \hOmega }{\rd \xi} \bigg],
\end{align*}
with the matrix elements
\begin{align*}
A_{ln} &= |m| \int_{-1}^{1} \rd \xi  \hPlm(\xi)\hOmega(\xi)\hPnm(\xi) , \\
B_{ln} &= 4 \int_{-1}^{1} \rd \xi  \hPlm(\xi) \bigg(\frac{1-\xi}{2}\bigg)^{1/2} \frac{\rd}{\rd \xi} \bigg[\bigg(\frac{1-\xi}{2}\bigg)^{5/4}\hPnm(\xi)\bigg], \\
C_{ln} &= |m| \int_{-1}^{1} \rd \xi \hPlm(\xi) \bigg(\frac{1-\xi}{2}\bigg)^{3/4}\bigg(\frac{1+\xi}{2}\bigg)^{-1}   \hPnm(\xi), \\
D_{ln} &= 4 \bigg(\frac{1}{2n+1}-\frac{\varepsilon}{3}  \bigg(\frac{1}{qx}\bigg)^{\frac{2}{3}} \bigg) \int_{-1}^{1} \rd \xi \hPlm(\xi)  \bigg(\frac{1-\xi}{2}\bigg)^{5/4} \bigg(\frac{1+\xi}{2}\bigg) 
\frac{\rd}{\rd \xi} \bigg[ \bigg(\frac{1-\xi}{2}\bigg)^{1/2}\hPnm(\xi) \bigg] ,\\
F_{ln} &=2 \int_{-1}^{1} \rd \xi  \hPlm(\xi)\hOmega(\xi)\hPnm(\xi) , \\
G_{ln} &= -|m| \bigg(\frac{1}{2n+1}-\frac{\varepsilon}{3}  \bigg(\frac{1}{qx}\bigg)^{\frac{2}{3}} \bigg)  \int_{-1}^{1} \rd \xi \hPlm(\xi)  \bigg(\frac{1-\xi}{2}\bigg)^{3/4} 
\hPnm(\xi)  ,\\
H_{ln} &=  \int_{-1}^{1} \rd \xi  \hPlm(\xi)\frac{\halpha^2(\xi)}{2\hOmega(\xi)}\hPnm(\xi) .
\end{align*}

Integrals $A_{ln}$ and  $F_{ln}$ are proportional. With the addition of $H_{ln}$, those 3 integrals must be computed numerically because of the non-trivial shift in their expression induced by the bulb potential. As for $B_{ln}$, $C_{ln}$, $D_{ln}$ and $G_{ln}$, their can be can expressed in terms of the two following integrals 
\begin{align}
\hI(l,n) &= \int_{-1}^{1} \rd \xi  \bigg(\frac{1-\xi}{2}\bigg)^{3/4} \hPlm(\xi)  \hPnm(\xi), \\
\hJ(l,n) &= \int_{-1}^{1} \rd \xi  \bigg(\frac{1-\xi}{2}\bigg)^{3/4}  \bigg(\frac{1+\xi}{2}\bigg)^{-1}\hPlm(\xi)  \hPnm(\xi) ,
\end{align}
as
\begin{align*}
B_{ln} &= \frac{1}{2} \bigg[ \sqrt{\frac{(2l+1)(l+m+1)(l-m+1)}{2l+3}} \hJ(l+1,n)+ \hJ(l,n)\\
&\quad \quad  -\sqrt{\frac{(2l+1)(l+m)(l-m)}{2l-1}} \hJ(l-1,n)\bigg] ,\\
C_{ln} &= m \hJ(l,n) ,\\
D_{ln} &= \frac{1}{2}  \bigg(\frac{1}{2n+1}-\frac{\varepsilon}{3}  \bigg(\frac{1}{qx}\bigg)^{\frac{2}{3}} \bigg)
\bigg[ -\sqrt{\frac{(2n+1)(n+m+1)(n-m+1)}{2n+3}} \hI(l,n+1) \\
&\quad \quad \quad \quad \quad \quad\quad \quad\quad \quad \quad \quad- \hI(l,n) \\
&\quad \quad \quad \quad \quad \quad\quad \quad\quad \quad \quad \quad  +\sqrt{\frac{(2n+1)(n+m)(n-m)}{2n-1}} \hI(l,n-1)\bigg] ,\\
G_{ln} &= -|m| \bigg(\frac{1}{2n+1}-\frac{\varepsilon}{3}  \bigg(\frac{1}{qx}\bigg)^{\frac{2}{3}} \bigg)  \hI(l,n) .
\end{align*}
where $\hI(l,n)$ and $\hJ(l,n)$ can be computed by recursion and using the symmetry $l\leftrightarrow n$. Defining
\begin{align}
\hIp(l,n) &= \int_{-1}^{1} \rd \xi \, \xi \bigg(\frac{1-\xi}{2}\bigg)^{3/4} \hPlm(\xi)  \hPnm(\xi), \\
\hJp(l,n) &= \int_{-1}^{1} \rd \xi \, \xi \bigg(\frac{1-\xi}{2}\bigg)^{3/4}  \bigg(\frac{1+\xi}{2}\bigg)^{-1}\hPlm(\xi)  \hPnm(\xi) ,
\end{align}
Starting from(Aoki79, A16)
\begin{align*}
\hI(l,n) &= \sqrt{\frac{(2n+1)(2n-1)}{(n+m)(n-m)}} \hIp(l,n-1)-\sqrt{\frac{(n+m-1)(n-m-1)(2n+1)}{(n+m)(n-m)(2n-3)}} \hI(l,n-2) ,\\
\hIp(l,n-1) &= \sqrt{\frac{(l+m+1)(l-m+1)}{(2l+1)(2l+3)}} \hI(l+1,n-1)+\sqrt{\frac{(l+m)(l-m)}{(2l+1)(2l-1)}} \hI(l-1,n-1) ,
\end{align*}
hence to compute until $l,n=m+N$, we need to compute $\hI(l',m)$ until $l'=m+2N$. By convention (for the recursion), we have set $\hI(l',n')=0$ for $l'<m$ or  $n'<m$.
We initialize with
\begin{align*}
\hI(l,m) &= \frac{l-3/4-m-1}{l+3/4+m+1} \sqrt{\frac{(l+m)(2l+1)}{(l-m)(2l-1)}} \hI(l-1,m) ,\\
\hI(m,m) &= 2^m \prod_{k=0}^{m} \frac{2k+1}{3/4 + m + 1 + k} .
\end{align*}

We proceed as follows:

- Compute the line $n=m$: $\hI(m,m)$, $\hI(m+1,m)$ , ... , $\hI(m+2N,m)$

- Complete the line $l=m$ by symmetry

- Compute the line $n=m+1$: $\hI(m+1,m+1)$, $\hI(m+2,m+1)$ , ... , $\hI(m+2N-1,m+1)$

- Complete the line $l=m+1$ by symmetry

- Compute the line $n=m+2$: $\hI(m+2,m+2)$, $\hI(m+3,m+2)$ , ... , $\hI(m+2N-2,m+2)$

- ...

- Compute the line $n=m+N-1$: $\hI(m+N-1,m+N-1)$, $\hI(m+N+1,m+N-1)$.

- Complete the line $l=m+N-1$ by symmetry

- Compute the line $n=m+N$: $\hI(m+N,m+N)$.

As for $\hJ$, let
\begin{align}
\hI_{\alpha}(l,m) &= \frac{(-1)^{l-m}(2m-1)!!2^{m+1}\Gamma(\alpha+1)\Gamma(\alpha+m+1)(l+m)!}{\Gamma(\alpha+1-l+m)\Gamma(\alpha+m+l+2)(l-m)!} 
\end{align}
with $\hI_{3/4}(l,m) =\hI(l,m)$. We can also compute it by recursion using the formulae
\begin{align*}
\hJ(l,n) &= \sqrt{\frac{(2n+1)(2n-1)}{(n+m)(n-m)}} \hJp(l,n-1)-\sqrt{\frac{(n+m-1)(n-m-1)(2n+1)}{(n+m)(n-m)(2n-3)}} \hJ(l,n-2) ,\\
\hJp(l,n-1) &= \sqrt{\frac{(l+m+1)(l-m+1)}{(2l+1)(2l+3)}} \hJ(l+1,n-1)+\sqrt{\frac{(l+m)(l-m)}{(2l+1)(2l-1)}} \hJ(l-1,n-1) ,
\end{align*}
hence to compute until $l,n=m+N$, we need to compute $\hI(l',m)$ until $l'=m+2N$. By convention (for the recursion), we have set $\hI(l',n')=0$ for $l'<m$ or  $n'<m$.
We initialize with
\begin{align*}
\hJ(l,m) &=\sqrt{\frac{(l-m)(l-m-1)(2l+1)}{(l+m)(l+m-1)(2l-3)}} \hJ(l-2,m)+ 4 \sqrt{\frac{(2m+1)(2l+1)(2l-1)}{2m(l+m)(l+m-1)}} \hI_{7/3}(l-1,m-1) ,\\
\hJ(m,m) &= \frac{2^m}{m} \prod_{k=1}^{m} \frac{2k+1}{3/4 + m + k} ,\\
\hJ(m+1,m) &= -\frac{7}{4}\frac{2^m\sqrt{2m+3}}{m} \prod_{k=0}^{m} \frac{2k+1}{3/4 + m + 1 +k} = -\frac{7\sqrt{2m+3}}{4(3/4+2m+1)} \hJ(m,m).
\end{align*}
We can compute the $\hI_{7/4}$ part by recursion. Indeed,
\begin{align}
\hI_{\alpha}(l,m) &= \frac{l-\alpha-m-1}{l+\alpha+m+1} \sqrt{\frac{(l+m)(2l+1)}{(l-m)(2l-1)}} \hI_{\alpha}(l-1,m) ,\\
\hI_{\alpha}(m,m) &= 2^m \prod_{k=0}^{m} \frac{2k+1}{\alpha + m + 1 + k} .
\end{align}

Hence, we need to compute beforehand the values $\hI_{\alpha}(l',m-1)$ for $l=m-1,...,m+2N-1$, and then apply the same process as for $\hI$. 

As for the numerical integral, we use a simple midpoint rule with $K$ points. Those integrals have the form
\begin{align*}
I_{ln} &= \int_{-1}^{1} \rd \xi \hPlm(\xi) \phi(\xi) \hPnm(\xi) \approx \frac{2}{K} \sum_{k=1}^{K} \hPlm(\xi_{k}) \phi(\xi_{k}) \hPnm(\xi_{k}),
\end{align*}
where $\xi_{k} = -1 + (2/K)(k-1/2)$. As we with to compute those elements for $m \leq l,n \leq m+N$, we have to compute the $\hPlm(\xi_{k})$ for $n=m,...,m+N$ and $k=1,...,K$. To that end, We compute before hand a table of the values $\{\hPlm(\xi_{k})\}_{(n,k)}$ and of the values $\{\phi(\xi_{k})\}_{k}$. The Legendre associated functions can be efficiently computed by using the Julia library "SphericalHarmomics", in which we use the function "computePlmcostheta($\theta,l_{\max},m$)" which compute $\hPlm(\cos(\theta))/\sqrt{\pi}$ for all $n=0,...,l_{\max}$ at a given $m$.


\end{document}
